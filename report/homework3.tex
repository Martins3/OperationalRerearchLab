\documentclass[fontset=none]{article} %取消CTeX的默认字体设置

\usepackage[UTF8]{ctex}
\usepackage{fancyhdr}
\usepackage{extramarks}
\usepackage{amsmath}
\usepackage{amsthm}
\usepackage{tikz}
% \usepackage[plain]{algorithm}
\usepackage{algpseudocode}
\usepackage{lmodern}
\usepackage{tikz}
\usepackage[linesnumbered,ruled,vlined]{algorithm2e}% 包

% 处理
\usepackage[english]{babel}
\usepackage{hyperref}
\hypersetup{
    colorlinks=true,
    linkcolor=blue,
    filecolor=magenta,      
    urlcolor=cyan,
}

\newtheorem{theorem}{Theorem}

%
% Basic Document Settings
%
\setCJKmainfont{SourceHanSansCN-Light}
\setlength{\parskip}{\baselineskip}%
\setlength{\parindent}{0pt}%



\topmargin=-0.45in
\evensidemargin=0in
\oddsidemargin=0in
\textwidth=6.5in
\textheight=9.0in
\headsep=0.25in

\linespread{1.1}

\pagestyle{fancy}
\lhead{\hmwkAuthorName}
\rhead{\firstxmark}
\lfoot{\lastxmark}
\cfoot{\thepage}

\renewcommand\headrulewidth{0.4pt}
\renewcommand\footrulewidth{0.4pt}

\setlength\parindent{0pt}

%
% Create Problem Sections
%

\newcommand{\enterProblemHeader}[1]{
    \nobreak\extramarks{}{Problem \arabic{#1} continued on next page\ldots}\nobreak{}
    \nobreak\extramarks{Problem \arabic{#1} (continued)}{Problem \arabic{#1} continued on next page\ldots}\nobreak{}
}

\newcommand{\exitProblemHeader}[1]{
    \nobreak\extramarks{Problem \arabic{#1} (continued)}{Problem \arabic{#1} continued on next page\ldots}\nobreak{}
    \stepcounter{#1}
    \nobreak\extramarks{Problem \arabic{#1}}{}\nobreak{}
}

\setcounter{secnumdepth}{0}
\newcounter{partCounter}
\newcounter{homeworkProblemCounter}
\setcounter{homeworkProblemCounter}{1}
\nobreak\extramarks{Section \arabic{homeworkProblemCounter}}{}\nobreak{}

%
% Homework Problem Environment
%
% This environment takes an optional argument. When given, it will adjust the
% problem counter. This is useful for when the problems given for your
% assignment aren't sequential. See the last 3 problems of this template for an
% example.
%
\newenvironment{homeworkProblem}[1][-1]{
    \ifnum#1>0
        \setcounter{homeworkProblemCounter}{#1}
    \fi
    \section{Section \arabic{homeworkProblemCounter}}
    \setcounter{partCounter}{1}
    \enterProblemHeader{homeworkProblemCounter}
}{
    \exitProblemHeader{homeworkProblemCounter}
}

%
% Homework Details
%   - Title
%   - Due date
%   - Class
%   - Section/Time
%   - Instructor
%   - Author
%

\newcommand{\hmwkTitle}{作业三}
\newcommand{\hmwkDueDate}{2020.6.24}
\newcommand{\hmwkClass}{高级算法}
\newcommand{\hmwkClassInstructor}{Professor cai}
\newcommand{\hmwkAuthorName}{\textbf{xueshi hu}}

%
% Title Page
%

\title{
    \vspace{2in}
    \textmd{\textbf{\hmwkClass:\ \hmwkTitle}}\\
    \vspace{3in}
}

\author{\hmwkAuthorName}
\date{}

\renewcommand{\part}[1]{\textbf{\large Part \Alph{partCounter}}\stepcounter{partCounter}\\}

%
% Various Helper Commands
%

% Useful for algorithms
\newcommand{\alg}[1]{\textsc{\bfseries \footnotesize #1}}

% For derivatives
\newcommand{\deriv}[1]{\frac{\mathrm{d}}{\mathrm{d}x} (#1)}

% For partial derivatives
\newcommand{\pderiv}[2]{\frac{\partial}{\partial #1} (#2)}

% Integral dx
\newcommand{\dx}{\mathrm{d}x}

% Alias for the Solution section header
\newcommand{\solution}{\textbf{\large Solution}}

% Probability commands: Expectation, Variance, Covariance, Bias
\newcommand{\E}{\mathrm{E}}
\newcommand{\Var}{\mathrm{Var}}
\newcommand{\Cov}{\mathrm{Cov}}
\newcommand{\Bias}{\mathrm{Bias}}

\begin{document}
\maketitle
\pagebreak

\begin{homeworkProblem}
问题描述: 请介绍一个问题(包括问题背景和定义),并介绍该问题的求解方案,包括(1)该问题到一般模型的编码,包括SAT,MaxSAT,CSP,MIP等;(2)针对该问题(原问题形式)设计一个局部搜索算法,需要给出基本数据结构和伪代码。此外,针对算法给出一定的理论分析,或给出某些实现要点,二者取其一(当然都做也可以,更好。)。以小论文(语言不限,可中文可英文)的形式提交,字数没有限制。
\end{homeworkProblem}

\begin{homeworkProblem}
图着色(Graph Coloring)问题指的是,对于一个无向图G(E,V),是否存在指定数量的颜色将所有的顶点
染色,其对于任意一个边,其两个顶点的颜色不同。其中取相同颜色的顶点可以构成独立集(Independent set)。

使用SAT对其进行编码,
上课讲解过一种编码方法是基于点的,简单复述一遍,加入共有$k$种颜色,$x_{ij}$表示顶点$i$取第$j$个颜色,由于每个顶点至少需要一个颜色,
构造子句$\lor_{j=1}^{k} x_{ij}$,相邻的边的颜色不可以冲突,对于边$mn$颜色$i$,构造子句$\lnot(x_{mi} \land x_{mi})$

下面提出一种新的基于边的编码方法:
\begin{itemize}   
  \item 对于边$i$,如果顶点编号较小的那个的颜色为$m$,另一个点的颜色为$n$,那么变量$x_{imn}$为真
  \item 每个点的颜色不能互相冲突,顶点$i$的两个临边不可以将其染色为$m$和$n$,可以构造$\lnot(M\land N)$,如果顶点$i$的临边的个数为$Nei$
    这种子句的数量为$\binom{2}{Nei}k(k-1)$,其中$\binom{2}{Nei}$表示选择的两个边可能性,$k(k-1)$表示一共可以选择的颜色种类。
  \item 所有的边最少存在一种染色模式,可以构造子句 $\lor_{m=1}^{k} \lor_{n=1}^{k} x_{imn}$
\end{itemize}

\end{homeworkProblem}

\begin{homeworkProblem}
下面考虑使用禁忌算法(Tabu)和混合进化(Hybrid Revolutionary)实现图着色算法,代码在

图着色问题,local search的基本方法是,首先对于所有的点
使用给定的颜色进行随机着色,然后冲突最多的点,将其颜色进行替换。
Tabu的思想在于让local search不要走回去,在图着色问题中间,对于刚刚修改颜色的点,不要在修改颜色。

下面简单分析其实现:
\begin{itemize}   
  \item 初始化数据结构,关键的为:
  \begin{itemize}   
    \item color\_conflict\_num : 二维数组,记录所有的顶点,使用不同颜色的冲突数量
    \item tabu\_tenure : 记录一个顶点发生修改的时间点
  \end{itemize}
  \item 每步,遍历整个 color\_conflict\_num 找到没有被禁忌的最大值,更新数据结构
\end{itemize}

相对朴素的 tabu search,存在两个简单的点需要加以说明:
\begin{itemize}   
  \item tabu tenure的实现,一种方法是,存储每一个顶点被tabu的时间,然后每一步减少1,直到不再被禁忌。但是,更加高效的实现方法是采用类似于时间戳的方法,记录当前的迭代次数,每次更新一个节点的时候,在数组tabu\_tenure中间记录修改的迭代次数,通过检查两者的差值来确定时间。
  \item 当被禁忌的节点可以导致冲突数量减少,而没有被禁忌的节点会导致冲突数量增加的时候,此时,无视禁忌。
\end{itemize}


初步测试结果在表格\ref{tabu:res}中。
\begin{table}
	\caption{Tabu Search Experiment Results}
	\centering
    \label{tabu:res}
    \begin{tabular}{l*{4}{c}r | p{1cm}}
      序号 & 数据 & 染色数 & 时间(s) & 禁忌步长  \\
    \hline
    1 & DSJC125.1 &5 &   0.013845 & 10     \\
    2 & DSJC250.1 &8 &   0.252603 & 10    \\
    3 & DSJC250.5 &28 & 95.3894   & 10     \\
    4 & DSJC250.9 &72 & NULL   & 10    \\
    5 & DSJC500.1 &13 & 0.187932   & 10    \\
    6 & DSJC500.1 &12 & NULL   & 10    \\
    7 & DSJC250.9 &73 &  NULL  & 10    \\
    8 & DSJC250.9 &100 &  0.055752  & 10    \\
    8 & DSJC250.9 &75 &  NULL  & 10    \\
    9 & DSJC250.9 &80 &  0.115334  & 10    \\
    10 & DSJC500.5 &51 & 6.62945   & 10    \\
    11 & DSJC500.5 &50 & 251.354   & 10    \\
    12 & DSJC500.5 &49 & NULL   & 10    \\
    13 & DSJC1000.1 &22 & 0.920005   & 10    \\
    14 & DSJC1000.1 &25 & 0.020005   & 10    \\
    15 & DSJC1000.1 &21 & 36.9668   & 10    \\
    \end{tabular}
\end{table}

\end{homeworkProblem}

\begin{homeworkProblem}
混合进化基于禁忌算法,其思想源自于生物,创建一个种群,杂交,去掉不良个体,如此迭代。
在图着色问题中间,首先利用禁忌算法计算出来一组解,从中间随机选择两个解构造的新的解,构造方法是交替选择其中一个解的染色节点,对于构造的解,运行禁忌算法,将得到的解替换掉最差的解。

说明一下"杂交"的过程:
\begin{enumerate}   
  \item 将禁忌算法的结果表示为二维数组solution,那么solution[i]表示颜色全部染成i的节点
  \item 对于"亲本",solution1和soution2,交替选择,构造子代offspring,也就是当$i\%==0$成立的时候,offspring[i]节点来自于solution1[1],否则来自于solution2[i]
  \item 上面的构造过程中间,当一个顶点被放到solution[i]之后,就不能放到其他的solution中间,这会导致一些节点无法染色,对于这些点,采取随机染色。
\end{enumerate}

初步测试结果在表格\ref{HEA:res}中。

\begin{table}
	\caption{HEA Search Experiment Results}
	\centering
    \label{HEA:res}
    \begin{tabular}{l*{4}{c}r | p{1cm}}
        序号 & 数据 & 染色数 & 时间 & 禁忌不长 & 种群大小  \\
    \hline
    1 & DSJC125.1 &5 &  0.063239  & 10 & 5    \\
    2 & DSJC250.1 &8 &   0.723432 & 10 & 5    \\
    3 & DSJC500.5 &52 &  10.3452  & 10 & 5    \\
    4 & DSJC500.5 &51 &  33.7604  & 10 & 5    \\
    5 & DSJC500.5 &50 & 109.607  & 10 & 5    \\
    5 & DSJC500.5 &49 & NULL  & 10 & 4    \\
    5 & DSJC500.5 &49 & NULL  & 10 & 10    \\
    6 & DSJC1000.1 &25 & 1.69291  & 10 & 5    \\
    \end{tabular}
\end{table}

两者实现的代码部署在github上: https://github.com/Martins3/OperationalRerearchLab
\end{homeworkProblem}


\end{document}
